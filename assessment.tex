% Options for packages loaded elsewhere
\PassOptionsToPackage{unicode}{hyperref}
\PassOptionsToPackage{hyphens}{url}
%
\documentclass[
]{article}
\usepackage{lmodern}
\usepackage{amssymb,amsmath}
\usepackage{ifxetex,ifluatex}
\ifnum 0\ifxetex 1\fi\ifluatex 1\fi=0 % if pdftex
  \usepackage[T1]{fontenc}
  \usepackage[utf8]{inputenc}
  \usepackage{textcomp} % provide euro and other symbols
\else % if luatex or xetex
  \usepackage{unicode-math}
  \defaultfontfeatures{Scale=MatchLowercase}
  \defaultfontfeatures[\rmfamily]{Ligatures=TeX,Scale=1}
\fi
% Use upquote if available, for straight quotes in verbatim environments
\IfFileExists{upquote.sty}{\usepackage{upquote}}{}
\IfFileExists{microtype.sty}{% use microtype if available
  \usepackage[]{microtype}
  \UseMicrotypeSet[protrusion]{basicmath} % disable protrusion for tt fonts
}{}
\makeatletter
\@ifundefined{KOMAClassName}{% if non-KOMA class
  \IfFileExists{parskip.sty}{%
    \usepackage{parskip}
  }{% else
    \setlength{\parindent}{0pt}
    \setlength{\parskip}{6pt plus 2pt minus 1pt}}
}{% if KOMA class
  \KOMAoptions{parskip=half}}
\makeatother
\usepackage{xcolor}
\IfFileExists{xurl.sty}{\usepackage{xurl}}{} % add URL line breaks if available
\IfFileExists{bookmark.sty}{\usepackage{bookmark}}{\usepackage{hyperref}}
\hypersetup{
  pdftitle={R Assessment},
  hidelinks,
  pdfcreator={LaTeX via pandoc}}
\urlstyle{same} % disable monospaced font for URLs
\usepackage[margin=1in]{geometry}
\usepackage{color}
\usepackage{fancyvrb}
\newcommand{\VerbBar}{|}
\newcommand{\VERB}{\Verb[commandchars=\\\{\}]}
\DefineVerbatimEnvironment{Highlighting}{Verbatim}{commandchars=\\\{\}}
% Add ',fontsize=\small' for more characters per line
\usepackage{framed}
\definecolor{shadecolor}{RGB}{248,248,248}
\newenvironment{Shaded}{\begin{snugshade}}{\end{snugshade}}
\newcommand{\AlertTok}[1]{\textcolor[rgb]{0.94,0.16,0.16}{#1}}
\newcommand{\AnnotationTok}[1]{\textcolor[rgb]{0.56,0.35,0.01}{\textbf{\textit{#1}}}}
\newcommand{\AttributeTok}[1]{\textcolor[rgb]{0.77,0.63,0.00}{#1}}
\newcommand{\BaseNTok}[1]{\textcolor[rgb]{0.00,0.00,0.81}{#1}}
\newcommand{\BuiltInTok}[1]{#1}
\newcommand{\CharTok}[1]{\textcolor[rgb]{0.31,0.60,0.02}{#1}}
\newcommand{\CommentTok}[1]{\textcolor[rgb]{0.56,0.35,0.01}{\textit{#1}}}
\newcommand{\CommentVarTok}[1]{\textcolor[rgb]{0.56,0.35,0.01}{\textbf{\textit{#1}}}}
\newcommand{\ConstantTok}[1]{\textcolor[rgb]{0.00,0.00,0.00}{#1}}
\newcommand{\ControlFlowTok}[1]{\textcolor[rgb]{0.13,0.29,0.53}{\textbf{#1}}}
\newcommand{\DataTypeTok}[1]{\textcolor[rgb]{0.13,0.29,0.53}{#1}}
\newcommand{\DecValTok}[1]{\textcolor[rgb]{0.00,0.00,0.81}{#1}}
\newcommand{\DocumentationTok}[1]{\textcolor[rgb]{0.56,0.35,0.01}{\textbf{\textit{#1}}}}
\newcommand{\ErrorTok}[1]{\textcolor[rgb]{0.64,0.00,0.00}{\textbf{#1}}}
\newcommand{\ExtensionTok}[1]{#1}
\newcommand{\FloatTok}[1]{\textcolor[rgb]{0.00,0.00,0.81}{#1}}
\newcommand{\FunctionTok}[1]{\textcolor[rgb]{0.00,0.00,0.00}{#1}}
\newcommand{\ImportTok}[1]{#1}
\newcommand{\InformationTok}[1]{\textcolor[rgb]{0.56,0.35,0.01}{\textbf{\textit{#1}}}}
\newcommand{\KeywordTok}[1]{\textcolor[rgb]{0.13,0.29,0.53}{\textbf{#1}}}
\newcommand{\NormalTok}[1]{#1}
\newcommand{\OperatorTok}[1]{\textcolor[rgb]{0.81,0.36,0.00}{\textbf{#1}}}
\newcommand{\OtherTok}[1]{\textcolor[rgb]{0.56,0.35,0.01}{#1}}
\newcommand{\PreprocessorTok}[1]{\textcolor[rgb]{0.56,0.35,0.01}{\textit{#1}}}
\newcommand{\RegionMarkerTok}[1]{#1}
\newcommand{\SpecialCharTok}[1]{\textcolor[rgb]{0.00,0.00,0.00}{#1}}
\newcommand{\SpecialStringTok}[1]{\textcolor[rgb]{0.31,0.60,0.02}{#1}}
\newcommand{\StringTok}[1]{\textcolor[rgb]{0.31,0.60,0.02}{#1}}
\newcommand{\VariableTok}[1]{\textcolor[rgb]{0.00,0.00,0.00}{#1}}
\newcommand{\VerbatimStringTok}[1]{\textcolor[rgb]{0.31,0.60,0.02}{#1}}
\newcommand{\WarningTok}[1]{\textcolor[rgb]{0.56,0.35,0.01}{\textbf{\textit{#1}}}}
\usepackage{graphicx,grffile}
\makeatletter
\def\maxwidth{\ifdim\Gin@nat@width>\linewidth\linewidth\else\Gin@nat@width\fi}
\def\maxheight{\ifdim\Gin@nat@height>\textheight\textheight\else\Gin@nat@height\fi}
\makeatother
% Scale images if necessary, so that they will not overflow the page
% margins by default, and it is still possible to overwrite the defaults
% using explicit options in \includegraphics[width, height, ...]{}
\setkeys{Gin}{width=\maxwidth,height=\maxheight,keepaspectratio}
% Set default figure placement to htbp
\makeatletter
\def\fps@figure{htbp}
\makeatother
\setlength{\emergencystretch}{3em} % prevent overfull lines
\providecommand{\tightlist}{%
  \setlength{\itemsep}{0pt}\setlength{\parskip}{0pt}}
\setcounter{secnumdepth}{-\maxdimen} % remove section numbering

\title{R Assessment}
\author{}
\date{\vspace{-2.5em}}

\begin{document}
\maketitle

In this Assessment we will analyze a dataset of ad impressions. Each
entry in this dataset is an impression id, information about the
customer and

\hypertarget{setting-up}{%
\subsection{Setting Up}\label{setting-up}}

\begin{enumerate}
\def\labelenumi{\arabic{enumi}.}
\item
  Install RStudio on your computer. Create a directory named
  \texttt{Assessment} somewhere on your computer and add an empty file
  named \texttt{answers.R} in it. Switch to that directory in RStudio
  and set the directory as your current working directory. Upload an
  image.
\item
  Clear all variables in the workspace. It's best to do this before any
  project.
\end{enumerate}

\begin{Shaded}
\begin{Highlighting}[]
\KeywordTok{rm}\NormalTok{(}\DataTypeTok{list=}\KeywordTok{ls}\NormalTok{())}
\end{Highlighting}
\end{Shaded}

\begin{enumerate}
\def\labelenumi{\arabic{enumi}.}
\setcounter{enumi}{2}
\tightlist
\item
  In this exercise we will need the \texttt{dplyr} library. Install this
  package if you don't have it, otherwise load it.
\end{enumerate}

\begin{Shaded}
\begin{Highlighting}[]
\KeywordTok{library}\NormalTok{(}\StringTok{'dplyr'}\NormalTok{)}
\end{Highlighting}
\end{Shaded}

\begin{verbatim}
## 
## Attaching package: 'dplyr'
\end{verbatim}

\begin{verbatim}
## The following objects are masked from 'package:stats':
## 
##     filter, lag
\end{verbatim}

\begin{verbatim}
## The following objects are masked from 'package:base':
## 
##     intersect, setdiff, setequal, union
\end{verbatim}

\begin{enumerate}
\def\labelenumi{\arabic{enumi}.}
\setcounter{enumi}{3}
\tightlist
\item
  Load the file \texttt{data1.csv} into the dataframe \texttt{ads}.
\end{enumerate}

\begin{Shaded}
\begin{Highlighting}[]
\NormalTok{ads <-}\StringTok{ }\KeywordTok{read.csv}\NormalTok{(}\StringTok{'data1.csv'}\NormalTok{)}
\end{Highlighting}
\end{Shaded}

\hypertarget{basic-data-analysis}{%
\subsection{Basic Data Analysis}\label{basic-data-analysis}}

\begin{enumerate}
\def\labelenumi{\arabic{enumi}.}
\setcounter{enumi}{3}
\tightlist
\item
  What are the column names of this dataset?
\end{enumerate}

\begin{Shaded}
\begin{Highlighting}[]
\KeywordTok{colnames}\NormalTok{(ads)}
\end{Highlighting}
\end{Shaded}

\begin{verbatim}
## [1] "id"         "date"       "channel"    "type"       "conversion"
## [6] "cost"       "revenue"
\end{verbatim}

\begin{enumerate}
\def\labelenumi{\arabic{enumi}.}
\setcounter{enumi}{4}
\tightlist
\item
  How may rows does the dataset have?
\end{enumerate}

\begin{Shaded}
\begin{Highlighting}[]
\KeywordTok{nrow}\NormalTok{(ads)}
\end{Highlighting}
\end{Shaded}

\begin{verbatim}
## [1] 1000
\end{verbatim}

\begin{enumerate}
\def\labelenumi{\arabic{enumi}.}
\setcounter{enumi}{5}
\tightlist
\item
  Give an initial summary of the dataset.
\end{enumerate}

\begin{Shaded}
\begin{Highlighting}[]
\KeywordTok{summary}\NormalTok{(ads)}
\end{Highlighting}
\end{Shaded}

\begin{verbatim}
##        id              date           channel    type      conversion   
##  Min.   :100000   Min.   :  2.00   display:262   A:329   Min.   :0.000  
##  1st Qu.:100250   1st Qu.: 88.75   email  :256   B:329   1st Qu.:0.000  
##  Median :100500   Median :176.00   print  :233   C:342   Median :0.000  
##  Mean   :100500   Mean   :176.78   search :249           Mean   :0.045  
##  3rd Qu.:100749   3rd Qu.:262.25                         3rd Qu.:0.000  
##  Max.   :100999   Max.   :364.00                         Max.   :1.000  
##       cost           revenue     
##  Min.   :0.0000   Min.   :1.000  
##  1st Qu.:0.4700   1st Qu.:1.520  
##  Median :0.9800   Median :2.020  
##  Mean   :0.9855   Mean   :2.013  
##  3rd Qu.:1.5000   3rd Qu.:2.520  
##  Max.   :2.0000   Max.   :2.990
\end{verbatim}

\begin{enumerate}
\def\labelenumi{\arabic{enumi}.}
\setcounter{enumi}{6}
\tightlist
\item
  Count up the total number of conversions over all ads.
\end{enumerate}

\begin{Shaded}
\begin{Highlighting}[]
\KeywordTok{sum}\NormalTok{(ads}\OperatorTok{$}\NormalTok{conversion)}
\end{Highlighting}
\end{Shaded}

\begin{verbatim}
## [1] 45
\end{verbatim}

\begin{enumerate}
\def\labelenumi{\arabic{enumi}.}
\setcounter{enumi}{7}
\tightlist
\item
  Filter our all impressions from the \texttt{display} channel that cost
  over \$1. Save the result in the datatable \texttt{expensive.ads}
\end{enumerate}

\begin{Shaded}
\begin{Highlighting}[]
\NormalTok{expensive.ads<-}\StringTok{ }\NormalTok{ads[(ads}\OperatorTok{$}\NormalTok{cost }\OperatorTok{>}\StringTok{ }\DecValTok{1}\NormalTok{)}\OperatorTok{&}\StringTok{ }\NormalTok{(ads}\OperatorTok{$}\NormalTok{channel}\OperatorTok{==}\StringTok{'display'}\NormalTok{),]}
\end{Highlighting}
\end{Shaded}

\begin{enumerate}
\def\labelenumi{\arabic{enumi}.}
\setcounter{enumi}{8}
\tightlist
\item
  For the ads in the previous question, what was the total spent?
\end{enumerate}

\begin{Shaded}
\begin{Highlighting}[]
\KeywordTok{sum}\NormalTok{(expensive.ads}\OperatorTok{$}\NormalTok{cost)}
\end{Highlighting}
\end{Shaded}

\begin{verbatim}
## [1] 189.46
\end{verbatim}

\hypertarget{datatable-manipulation}{%
\subsection{Datatable Manipulation}\label{datatable-manipulation}}

\begin{enumerate}
\def\labelenumi{\arabic{enumi}.}
\setcounter{enumi}{8}
\item
  Some impression id's are duplicated. Remove the duplicates.
\item
  Add a column which for each ad represents the profit. Compute the
  total profit.
\end{enumerate}

\begin{Shaded}
\begin{Highlighting}[]
\NormalTok{ads}\OperatorTok{$}\NormalTok{profit <-}\StringTok{ }\NormalTok{ads}\OperatorTok{$}\NormalTok{revenue }\OperatorTok{-}\StringTok{ }\NormalTok{ads}\OperatorTok{$}\NormalTok{cost}
\KeywordTok{sum}\NormalTok{(ads}\OperatorTok{$}\NormalTok{profit)}
\end{Highlighting}
\end{Shaded}

\begin{verbatim}
## [1] 1027.61
\end{verbatim}

\begin{enumerate}
\def\labelenumi{\arabic{enumi}.}
\setcounter{enumi}{10}
\tightlist
\item
  Create a new dataframe that for each channel shows the total
  impressions, total conversions, and total cost.
\end{enumerate}

\begin{Shaded}
\begin{Highlighting}[]
\NormalTok{channel.per <-}\StringTok{ }\NormalTok{ads }\OperatorTok\StringTok{ }\KeywordTok{group_by}\NormalTok{(channel) }\OperatorTok\StringTok{ }\KeywordTok{summarise}\NormalTok{(}\DataTypeTok{total.impressions=}\KeywordTok{n}\NormalTok{(), }\DataTypeTok{total.conversions=}\KeywordTok{sum}\NormalTok{(conversion), }\DataTypeTok{total.cost=}\KeywordTok{sum}\NormalTok{(cost))}
\end{Highlighting}
\end{Shaded}

\begin{verbatim}
## `summarise()` ungrouping output (override with `.groups` argument)
\end{verbatim}

\begin{Shaded}
\begin{Highlighting}[]
\KeywordTok{print}\NormalTok{(channel.per)}
\end{Highlighting}
\end{Shaded}

\begin{verbatim}
## # A tibble: 4 x 4
##   channel total.impressions total.conversions total.cost
##   <fct>               <int>             <int>      <dbl>
## 1 display               262                12       253.
## 2 email                 256                15       242.
## 3 print                 233                 6       230.
## 4 search                249                12       261.
\end{verbatim}

\begin{enumerate}
\def\labelenumi{\arabic{enumi}.}
\setcounter{enumi}{11}
\tightlist
\item
  Something about merging?
\end{enumerate}

\hypertarget{modeling}{%
\subsection{Modeling}\label{modeling}}

\begin{enumerate}
\def\labelenumi{\arabic{enumi}.}
\setcounter{enumi}{9}
\tightlist
\item
  Something about lm and glm
\end{enumerate}

\hypertarget{plotting}{%
\subsection{Plotting}\label{plotting}}

\end{document}
